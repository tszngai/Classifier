\documentclass{article}

% if you need to pass options to natbib, use, e.g.:
%     \PassOptionsToPackage{numbers, compress}{natbib}
% before loading neurips_2018

% ready for submission
% \usepackage{neurips_2018}

% to compile a preprint version, e.g., for submission to arXiv, add add the
% [preprint] option:
%     \usepackage[preprint]{neurips_2018}

% to compile a camera-ready version, add the [final] option, e.g.:
     \usepackage[final]{nips_2018}

% to avoid loading the natbib package, add option nonatbib:
%     \usepackage[nonatbib]{neurips_2018}

\usepackage[utf8]{inputenc} % allow utf-8 input
\usepackage[T1]{fontenc}    % use 8-bit T1 fonts
\usepackage{hyperref}       % hyperlinks
\usepackage{url}            % simple URL typesetting
\usepackage{booktabs}       % professional-quality tables
\usepackage{amsfonts}       % blackboard math symbols
\usepackage{nicefrac}       % compact symbols for 1/2, etc.
\usepackage{microtype}      % microtypography

\title{Formatting instructions for NeurIPS 2018}

% The \author macro works with any number of authors. There are two commands
% used to separate the names and addresses of multiple authors: \And and \AND.
%
% Using \And between authors leaves it to LaTeX to determine where to break the
% lines. Using \AND forces a line break at that point. So, if LaTeX puts 3 of 4
% authors names on the first line, and the last on the second line, try using
% \AND instead of \And before the third author name.

\author{%
  Yuan Sun\\
  Department of Computer Engineering\\
  University of British Columbia\\
  Vancouver, BC\\
  \texttt{round.sun@alumni.ubc.ca}\\
  \And
  Yixuen Ji\\
  Department of Computer Engineering\\
  University of British Columbia\\
  Vancouver, BC\\
  \texttt{jiyixuan@ece.ubc.ca}\\
  \AND
  Jay Fu\\
  Department of Computer Engineering\\
  University of British Columbia\\
  Vancouver, BC\\
  \texttt{jay.fu@alumni.ubc.ca}\\
}

\begin{document}
% \nipsfinalcopy is no longer used

\maketitle

\begin{abstract}
  This is a good abstract.
\end{abstract}

\section{Introduction}

This is instruction section.

\subsection{sub}

This is a good sub section.

\subsection{sub}

This is a good sub section.

\section{Related Work}
\label{gen_inst}

This is related work section.

\section{Dataset and Settings}
\label{headings}

This is Dataset and Settings section.

\subsection{sub}

This is a good sub section.

\subsubsection{subsub}

This is a good subsub section.

\section{Experiments}
\label{others}

This is experiments section.

\subsection{sub}

This is a good sub section.

\subsection{sub}

This is a good sub section.

\subsection{sub}

This is a good sub section.

\subsection{sub}

This is a good sub section.

\section{Discussion}

This is discussion section.

\subsection{sub}

This is a good sub section.

\section*{References}

References follow the acknowledgments. Use unnumbered first-level heading for
the references. Any choice of citation style is acceptable as long as you are
consistent. It is permissible to reduce the font size to \verb+small+ (9 point)
when listing the references. {\bf Remember that you can use more than eight
  pages as long as the additional pages contain \emph{only} cited references.}
\medskip

\small

[1] Alexander, J.A.\ \& Mozer, M.C.\ (1995) Template-based algorithms for
connectionist rule extraction. In G.\ Tesauro, D.S.\ Touretzky and T.K.\ Leen
(eds.), {\it Advances in Neural Information Processing Systems 7},
pp.\ 609--616. Cambridge, MA: MIT Press.

[2] Bower, J.M.\ \& Beeman, D.\ (1995) {\it The Book of GENESIS: Exploring
  Realistic Neural Models with the GEneral NEural SImulation System.}  New York:
TELOS/Springer--Verlag.

[3] Hasselmo, M.E., Schnell, E.\ \& Barkai, E.\ (1995) Dynamics of learning and
recall at excitatory recurrent synapses and cholinergic modulation in rat
hippocampal region CA3. {\it Journal of Neuroscience} {\bf 15}(7):5249-5262.

\end{document}
